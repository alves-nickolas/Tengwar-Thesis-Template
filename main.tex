\documentclass[a4paper, 11pt]{tengwarthesis}
%% Class Options:
% englishonly: apart from user specified Portuguese text, everything is typeset in English
% noglos: turns off glossary (abbreviations)
% notoc: turns off table of contents
% nohyper: turns off colored hyperlinks
% nocatcard: turns off catalographic card
% noshowkeys: turns off showkeys package (https://www.ctan.org/pkg/showkeys)

% If you don't want to use some of the fields below, just comment out its line or delete it. For example, if you don't want to link your ORCiD, just remove the \orcid command and the document should automatically understand you don't want it at all. WRITING \orcid{} AND KEEPING THE COMMAND IN THERE MIGHT LEAD TO ISSUES. 

%% Basic Document Info
\title{A Thesis with a Possibly Very Long and Interesting Title} % Title
\subtitle{Perhaps It Also Has a Subtitle} % Subtitle
\author{Given Name}{Family Name} % Author's Name \author{given name}{family name}
\orcid{https://orcid.org/XXXX-XXXX-XXXX-XXXX} % Author's ORCiD id
\date{\today} % Full date
\version{\today} % Current version (for drafts)

%% Cover and Title Page Information
\universityname[University's Portuguese Name]{University's English Name} % Full university name (optional argument for Portuguese)
\programname[Program's Portuguese Name]{Program's English Name} % Full program name (optional argument for Portuguese)
\presentationnote{Thesis presented to the Graduate Program in Physics at the University as partial requirement to obtain the degree of Master in Physics.} % Dissertation presented in partial fulfillment...
\advisor[Prof. Dr.]{Given Name}{Family Name} % Advisor \advisor[titles]{given name}{family name}
\coadvisor[Prof.]{John}{Doe} % Co-advisor \coadvisor[titles]{given name}{family name}
\location[State]{City}{Country} % Full location \location[state]{city}{country}
\defenseyear{2023} %Defense year

%% Catalographic Card Information (configured in Portuguese)
\thesistype{Thesis (MSc)} % Dissertação (Mestrado) or Tese (Doutorado)

%statement of compliance with committee observations
\declaracaodeatendimento{Este exemplar foi revisado e alterado em relação à versão original, de acordo com as observações levantadas pela banca examinadora no dia da defesa, sob responsabilidade única do autor e com a anuência do orientador.} % Portuguese version

\statementofcompliance{This copy was revised and altered with respect to the original version, according with the observations raised by the examining committee on the defense day, under the author's sole responsibility and with the advisor's consent.} % English version

%Signature Sheet
\signsheet{\includegraphics[height=\pageheight,width=\pagewidth]{example-image-plain}} %plain grey image as an example

%Funding Statement
\fundingstatement{This study was financed in part by the Funding Agency.}

\dedication{To someone}

%\epigraph{Heartfelt sentence}{Author} %single epigraph
\epigraphlist{\qitem{First epigraph}{First author}
\qitem{Second epigraph}{Second author}
\qitem{Third epigraph}{Third author}}

\resumo{Abstract in portuguese}
\palavraschave{first keyword in Portuguese}{second keyword in Portuguese}{third keyword in Portuguese} % up to five, all in curly bracket

\abstract{Abstract in English}
\keywords{first keyword in English}{second keyword in English}{third keyword in English} % up to five, all in curly bracket


\acknowledgments{\begin{epigraphs}
    \qitem{Some phrase}{Some author}
    \qitem{Other phrase}{Other author}
\end{epigraphs}

I'd like to acknowledge some people\ldots}

\preface{\kant[1]}

\begin{document}

\maketitle

\includecomment{chap1} % code in chap1 included
\excludecomment{chap2} % code in chap2 automatically commented out

\includecomment{app1} % code in app1 included

\begin{chap1}
\chapter{Chapter Title}
    \chapabstract{This is a brief, abstract-like, description of the chapter.}
    \begin{epigraphs}
        \qitem{Some inspirational quote for this chapter.}{The author of the quote.}
        \qitem{A second quote.}{The other author.}
    \end{epigraphs}
    \label{chap: chap-1}
    
    \kant[1-3]

Let us also cite some references: \textcite{aguiaralves2023NonperturbativeAspectsQuantum,codello2010NovelFunctionalRenormalization,donoghue2012EffectiveFieldTheory,hawking1975ParticleCreationBlack,khavkine2015AlgebraicQFTCurved,wald1994QuantumFieldTheory,weinberg1995QuantumTheoryFieldsSeries}.

We can also discuss some abbreviations, such as \gls{qftcs}, \gls{gr}, and \gls{qg}. These abbreviations will only occur as abbreviations if we mention them twice in the same chapter. Otherwise, they will just be fully expanded. See: \gls{qftcs} and \gls{qg}.
    \asterism
\end{chap1}

\begin{chap2}
\chapter{Another Chapter}
    \chapabstract{Description of the second chapter.}
    \epigraph{Some inspirational quote.}{Some famous author.}
    \label{chap: chap-2}
    
    \kant[4-6]
    \asterism
\end{chap2}

\makeback % ends main matter and starts appendices and etc

\begin{app1}
\chapter{Appendix Title}
    %\epigraph{Toto, I have a feeling we're not in Kansas anymore.}{Dorothy (Judy Garland) in \emph{The Wizard of Oz}.} % could have an epigraph and etc
    \label{app: app-1}
    
    \kant[7-9]
    \asterism
\end{app1}

\makefinal

\end{document}
